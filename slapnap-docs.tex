\documentclass[]{article}
\usepackage{lmodern}
\usepackage{amssymb,amsmath}
\usepackage{ifxetex,ifluatex}
\usepackage{fixltx2e} % provides \textsubscript
\ifnum 0\ifxetex 1\fi\ifluatex 1\fi=0 % if pdftex
  \usepackage[T1]{fontenc}
  \usepackage[utf8]{inputenc}
\else % if luatex or xelatex
  \ifxetex
    \usepackage{mathspec}
  \else
    \usepackage{fontspec}
  \fi
  \defaultfontfeatures{Ligatures=TeX,Scale=MatchLowercase}
\fi
% use upquote if available, for straight quotes in verbatim environments
\IfFileExists{upquote.sty}{\usepackage{upquote}}{}
% use microtype if available
\IfFileExists{microtype.sty}{%
\usepackage[]{microtype}
\UseMicrotypeSet[protrusion]{basicmath} % disable protrusion for tt fonts
}{}
\PassOptionsToPackage{hyphens}{url} % url is loaded by hyperref
\usepackage[unicode=true]{hyperref}
\hypersetup{
            pdftitle={slapnap: Super LeArner Prediction of NAb Panels},
            pdfauthor={David Benkeser, Brian D. Williamson, Craig A. Magaret, Bhavesh R. Borate, Peter B. Gilbert},
            pdfborder={0 0 0},
            breaklinks=true}
\urlstyle{same}  % don't use monospace font for urls
\usepackage[margin=1in]{geometry}
\usepackage{natbib}
\bibliographystyle{plainnat}
\usepackage{color}
\usepackage{fancyvrb}
\newcommand{\VerbBar}{|}
\newcommand{\VERB}{\Verb[commandchars=\\\{\}]}
\DefineVerbatimEnvironment{Highlighting}{Verbatim}{commandchars=\\\{\}}
% Add ',fontsize=\small' for more characters per line
\usepackage{framed}
\definecolor{shadecolor}{RGB}{248,248,248}
\newenvironment{Shaded}{\begin{snugshade}}{\end{snugshade}}
\newcommand{\KeywordTok}[1]{\textcolor[rgb]{0.13,0.29,0.53}{\textbf{#1}}}
\newcommand{\DataTypeTok}[1]{\textcolor[rgb]{0.13,0.29,0.53}{#1}}
\newcommand{\DecValTok}[1]{\textcolor[rgb]{0.00,0.00,0.81}{#1}}
\newcommand{\BaseNTok}[1]{\textcolor[rgb]{0.00,0.00,0.81}{#1}}
\newcommand{\FloatTok}[1]{\textcolor[rgb]{0.00,0.00,0.81}{#1}}
\newcommand{\ConstantTok}[1]{\textcolor[rgb]{0.00,0.00,0.00}{#1}}
\newcommand{\CharTok}[1]{\textcolor[rgb]{0.31,0.60,0.02}{#1}}
\newcommand{\SpecialCharTok}[1]{\textcolor[rgb]{0.00,0.00,0.00}{#1}}
\newcommand{\StringTok}[1]{\textcolor[rgb]{0.31,0.60,0.02}{#1}}
\newcommand{\VerbatimStringTok}[1]{\textcolor[rgb]{0.31,0.60,0.02}{#1}}
\newcommand{\SpecialStringTok}[1]{\textcolor[rgb]{0.31,0.60,0.02}{#1}}
\newcommand{\ImportTok}[1]{#1}
\newcommand{\CommentTok}[1]{\textcolor[rgb]{0.56,0.35,0.01}{\textit{#1}}}
\newcommand{\DocumentationTok}[1]{\textcolor[rgb]{0.56,0.35,0.01}{\textbf{\textit{#1}}}}
\newcommand{\AnnotationTok}[1]{\textcolor[rgb]{0.56,0.35,0.01}{\textbf{\textit{#1}}}}
\newcommand{\CommentVarTok}[1]{\textcolor[rgb]{0.56,0.35,0.01}{\textbf{\textit{#1}}}}
\newcommand{\OtherTok}[1]{\textcolor[rgb]{0.56,0.35,0.01}{#1}}
\newcommand{\FunctionTok}[1]{\textcolor[rgb]{0.00,0.00,0.00}{#1}}
\newcommand{\VariableTok}[1]{\textcolor[rgb]{0.00,0.00,0.00}{#1}}
\newcommand{\ControlFlowTok}[1]{\textcolor[rgb]{0.13,0.29,0.53}{\textbf{#1}}}
\newcommand{\OperatorTok}[1]{\textcolor[rgb]{0.81,0.36,0.00}{\textbf{#1}}}
\newcommand{\BuiltInTok}[1]{#1}
\newcommand{\ExtensionTok}[1]{#1}
\newcommand{\PreprocessorTok}[1]{\textcolor[rgb]{0.56,0.35,0.01}{\textit{#1}}}
\newcommand{\AttributeTok}[1]{\textcolor[rgb]{0.77,0.63,0.00}{#1}}
\newcommand{\RegionMarkerTok}[1]{#1}
\newcommand{\InformationTok}[1]{\textcolor[rgb]{0.56,0.35,0.01}{\textbf{\textit{#1}}}}
\newcommand{\WarningTok}[1]{\textcolor[rgb]{0.56,0.35,0.01}{\textbf{\textit{#1}}}}
\newcommand{\AlertTok}[1]{\textcolor[rgb]{0.94,0.16,0.16}{#1}}
\newcommand{\ErrorTok}[1]{\textcolor[rgb]{0.64,0.00,0.00}{\textbf{#1}}}
\newcommand{\NormalTok}[1]{#1}
\usepackage{longtable,booktabs}
% Fix footnotes in tables (requires footnote package)
\IfFileExists{footnote.sty}{\usepackage{footnote}\makesavenoteenv{long table}}{}
\usepackage{graphicx,grffile}
\makeatletter
\def\maxwidth{\ifdim\Gin@nat@width>\linewidth\linewidth\else\Gin@nat@width\fi}
\def\maxheight{\ifdim\Gin@nat@height>\textheight\textheight\else\Gin@nat@height\fi}
\makeatother
% Scale images if necessary, so that they will not overflow the page
% margins by default, and it is still possible to overwrite the defaults
% using explicit options in \includegraphics[width, height, ...]{}
\setkeys{Gin}{width=\maxwidth,height=\maxheight,keepaspectratio}
\IfFileExists{parskip.sty}{%
\usepackage{parskip}
}{% else
\setlength{\parindent}{0pt}
\setlength{\parskip}{6pt plus 2pt minus 1pt}
}
\setlength{\emergencystretch}{3em}  % prevent overfull lines
\providecommand{\tightlist}{%
  \setlength{\itemsep}{0pt}\setlength{\parskip}{0pt}}
\setcounter{secnumdepth}{5}
% Redefines (sub)paragraphs to behave more like sections
\ifx\paragraph\undefined\else
\let\oldparagraph\paragraph
\renewcommand{\paragraph}[1]{\oldparagraph{#1}\mbox{}}
\fi
\ifx\subparagraph\undefined\else
\let\oldsubparagraph\subparagraph
\renewcommand{\subparagraph}[1]{\oldsubparagraph{#1}\mbox{}}
\fi

% set default figure placement to htbp
\makeatletter
\def\fps@figure{htbp}
\makeatother


\title{\texttt{slapnap}: Super LeArner Prediction of NAb Panels}
\author{David Benkeser, Brian D. Williamson, Craig A. Magaret, Bhavesh R.
Borate, Peter B. Gilbert}
\date{May 22, 2020}

\begin{document}
\maketitle

{
\setcounter{tocdepth}{2}
\tableofcontents
}
\section*{Welcome}\label{welcome}
\addcontentsline{toc}{section}{Welcome}

The \href{https://hub.docker.com/r/slapnap/slapnap}{\texttt{slapnap}}
container is a tool for using the Compile, Analyze and Tally NAb Panels
(CATNAP) database to develop predictive models of HIV-1 neutralization
sensitivity to one or several broadly neutralizing antibodies (bNAbs).

\begin{center}\includegraphics[width=0.7\linewidth]{gp120} \end{center}\begin{center}
Crystal structure of HIV-1 gp120 glycoprotein. Highlighted residues
indicating sites most-predictive of VRC01 neutralization resistance.
{[}@magaret2019prediction{]}
\end{center}

In its simplest form, \texttt{slapnap} can be used simply to access and
format data from CATNAP in a way that is usable for machine learning
analysis. However, the tool also offers fully automated and customizable
machine learning analyses based on up to five different neutralization
endpoints, complete with automated report generation to summarize
results and identify the most predictive features.

This document serves as the user manual for the \texttt{slapnap}
container. Here, we describe everything needed to utilize the
\texttt{slapnap} container and understand its output. The documentation
is organized into the following sections:

\begin{itemize}
\tightlist
\item
  Section \ref{sec:docker} provides a brief overview of Docker,
  including information on installing Docker and downloading the
  \texttt{slapnap} container.
\item
  Section \ref{sec:catnap} provides a brief overview of the CATNAP
  database and the specifics of how the data were accessed to build the
  \texttt{slapnap} container.
\item
  Section \ref{sec:runningcontainer} provides a detailed description of
  how to make calls to the slapnap repository, including descriptions of
  all options that are available.
\item
  Section \ref{sec:examples} includes several example calls to the
  \texttt{slapnap} container and descriptions of their output.
\item
  Section \ref{sec:data} provides a description of the data set created
  in the \texttt{slapnap} container.
\item
  Section \ref{sec:methods} provides an overview of the methodology that
  is used in within the \texttt{slapnap} analysis.
\end{itemize}

If you have any issues or questions about using \texttt{slapnap}, please
\href{https://github.com/benkeser/slapnap/issues}{file an issue} on
GitHub.

\section{Docker}\label{sec:docker}

Docker is a free platform for building containers. Containers are
standard units of software that package code and all its dependencies,
so that the code can be executed reliably irrespective of computing
environment. The \texttt{slapnap} tool relies on machine learning
implemented in the \texttt{R} language and relies on several packages.
Achieving full reproducibility for such analyses is challenging in that
it requires synchronization across the specific version of \texttt{R}
and dependent packages. In other words, two users running two versions
of \texttt{R} or two versions of the same \texttt{R} package may arrive
at different output when running the same code. Containerization ensures
that this does not happen. Any two runs of \texttt{slapnap} with the
same input options will yield the same output every time.

\href{https://docs.docker.com/docker-for-windows/install/}{Installing
Docker} is necessary for running the \texttt{slapnap} tool. While it is
not necessary for execution of the \texttt{slapnap} container, readers
interested in learning more about Docker should consult the
\href{https://docs.docker.com/get-started/}{Docker documentation} for
information about getting started using Docker.

Once Docker has been installed on your local computer, you can download
\texttt{slapnap} using the following command.

\begin{Shaded}
\begin{Highlighting}[]
\ExtensionTok{docker}\NormalTok{ pull slapnap/slapnap}
\end{Highlighting}
\end{Shaded}

This command pulls the image from
\href{https://hub.docker.com/}{DockerHub}. Once the image has been
downloaded, we are ready to learn about how to execute \texttt{slapnap}
jobs. The next section contains information on the source data used by
\texttt{slapnap}. Users familiar with the CATNAP data may wish to skip
directly to Section \ref{sec:opts}.

\section{CATNAP Database}\label{sec:catnap}

The
\href{https://www.hiv.lanl.gov/components/sequence/HIV/neutralization/index.html}{CATNAP
database} is a web server hosted by Los Alamos National Laboratory
\citep{yoon2015catnap}. The database integrates antibody neutralization
and HIV-1 sequence data from published studies. Neutralization is
measured in terms of half maximal inhibitory concentration (IC\(_50\))
and 80\% inhibitory concentration (IC\(_80\)). These measures of
neutralization against HIV envelope pseudoviruses are available for many
broadly neutralizing antibodies (bNAbs) and for some combination bNAbs.
Also available on each pseudovirus are amino acid (AA) sequence features
for the gp160 protein. These are detailed in Section \ref{sec:data}.

During each build of the \texttt{slapnap} container, all raw data are
downloaded from CATNAP. At run time, the relevant data are selected and
processed into a format that is amenable for predictive machine learning
analyses. The CATNAP data are updated periodically. To check the date
the raw data were pulled from CATNAP to \texttt{slapnap}, you can check
the date of the \texttt{latest} build
\href{https://hub.docker.com/repository/registry-1.docker.io/slapnap/slapnap/tags?page=1}{here}.

\section{\texorpdfstring{Running the \texttt{slapnap}
container}{Running the slapnap container}}\label{sec:runningcontainer}

To run the \texttt{slapnap} container, we make use of the
\href{https://docs.docker.com/engine/reference/run/}{\texttt{docker\ run}}
command. Note that administrator (\texttt{sudo}) privileges are needed
to execute this command.

There are several options that are necessary to include in this command
to control the behavior of \texttt{slapnap}. These are discussed in
separate subsections below.

\subsection{Mounting a local directory}\label{sec:mounting}

At the end of a \texttt{slapnap} run, user-specified output will be
saved (see option \texttt{return} in Section \ref{sec:opts}). To
retrieve these files from inside the container, we
\href{https://docs.docker.com/storage/bind-mounts/}{\emph{mount}} a
local directory to an output directory (\texttt{/home/out/}) in the
container using the \texttt{-v} option. That is, all files in the
mounted local directory will be visible to programs running inside the
container and any items saved to the output directory in the container
(file path in the container \texttt{/home/out/}) will be available in
the mounted directory.

Suppose \texttt{/path/to/local/dir} is the file path on a local computer
in which we wish to save the output files from a \texttt{slapnap} run. A
\texttt{docker\ run} of \texttt{slapnap} would include the option
\texttt{-v\ /path/to/local/dir:/home/out}. After a run completes, the
requested output should be viewable in \texttt{/path/to/local/dir}. See
Section \ref{sec:examples} for full syntax.

\subsection{\texorpdfstring{\texttt{slapnap}
options}{slapnap options}}\label{sec:opts}

The user has control over many aspects of \texttt{slapnap}'s behavior.
These options are passed in using the \texttt{-e} option\footnote{This
  sets an environment variable in the container environment. These
  variables are accessed by the various \texttt{R} and \texttt{bash}
  scripts in the container to dictate how the container executes code.}.
Semi-colon separated strings are used to set options. For example, to
provide input for the option \texttt{option\_name}, we would used
\texttt{-e\ option\_name="a;semi-colon;separated;string"}. Note that
there are no spaces between the option name and its value and no spaces
after semi-colons in the separated list. See Section \ref{sec:examples}
for full syntax.

Each description below lists the default value that is assumed if the
option is not specified. Note that many of the default values are chosen
simply so that na\{"i\}ve calls to \texttt{slapnap} compile quickly.
Proper values should be determined based on scientific context.

\textbf{-e options for \texttt{slapnap}}

\begin{itemize}
\tightlist
\item
  \textbf{\texttt{nab}}: A semicolon-separated list of bNAbs (default =
  ``\texttt{VRC01}''). A list of possible bNAbs can be found
  \href{https://www.hiv.lanl.gov/components/sequence/HIV/neutralization/main.comp}{here}.
  If multiple bNAbs are listed, it is assumed that the analysis should
  be of estimated \texttt{outcomes} for a combination of bNAbs (see
  Section \ref{sec:outcomedefs} for details on how estimated outcomes
  for multiple bNAbs are computed).
\item
  \textbf{\texttt{outcomes}}: A semicolon-separated string of outcomes
  to include in the analysis. Possible values are \texttt{"ic50"}
  (included in default), \texttt{"ic80"}, \texttt{"iip"},
  \texttt{"sens1"} (included in default), \texttt{"sens2"} (for
  definitions of outcomes see Section \ref{sec:outcomedefs}). Of note,
  if only a single \texttt{nab} is included, then \texttt{sens1} and
  \texttt{sens2} are the same.
\item
  \textbf{\texttt{learners}}: A semicolon-separated string of machine
  learning algorithms to include in the analysis. Possible values
  include \texttt{"rf"} (random forest, default), \texttt{"xgboost"}
  (eXtreme gradient boosting), and \texttt{"lasso"} (elastic net). If
  more than one algorithm is included, then it is assumed that a
  cross-validated-based ensemble (i.e., a super learner) is desired (see
  Section \ref{sec:sldetails}).
\item
  \textbf{\texttt{cvtune}}: A boolean string (i.e., either
  \texttt{"TRUE"} or \texttt{"FALSE"} {[}default{]}) indicating whether
  the \texttt{learners} should be tuned using cross-validation and a
  small grid search. Defaults to \texttt{"FALSE"}. If multiple
  \texttt{learners} are specified, then the super learner ensemble
  includes three versions of each of the requested \texttt{learners}
  with different tuning parameters.
\item
  \textbf{\texttt{cvperf}}: A boolean string (i.e., either
  \texttt{"TRUE"} or \texttt{"FALSE"} {[}default{]}) indicating whether
  the \texttt{learners} performance should be evaluated using
  cross-validation. If \texttt{cvtune="TRUE"} or \texttt{learners}
  includes multiple algorithms, then nested cross-validation is used to
  evaluate the performance of the cross-validation-selected best value
  of tuning parameters for the specified algorithm or the super learner,
  respectively.
\item
  \textbf{\texttt{nfolds}}: A numeric string indicating the number of
  folds to use in cross-validation procedures (default = \texttt{"2"}).
\item
  \textbf{\texttt{importance\_grp}}: A semicolon-separated string
  indicating which group-level variable importance measures should be
  computed. Possible values are none \texttt{""} (default), marginal
  \texttt{"marg"}, conditional \texttt{"cond"}. See Section
  \ref{sec:biolimp} for details on these measures.
\item
  \textbf{\texttt{importance\_ind}}: A semicolon-separated string
  indicating which individual-level variable importance measures should
  be computed. Possible values are none \texttt{""} (default),
  learner-level \texttt{"pred"}, marginal \texttt{"marg"} and
  conditional \texttt{"cond"}. The latter two take significant
  computation time to compute.
\item
  \textbf{\texttt{report\_name}}: A string indicating the desired name
  of the output report (default =
  \texttt{report\_{[}\_-separated\ list\ of\ nabs{]}\_{[}date{]}.html}).
\item
  \textbf{\texttt{return}}: A semicolon-separated string of the desired
  output. Possible values are \texttt{"report"} (default),
  \texttt{"learner"} for the trained algorithm, \texttt{"data"} for the
  analysis dataset, \texttt{"figures"} for all figures from the report,
  and \texttt{"vimp"} for variable importance objects.
\end{itemize}

\subsection{Interactive sessions}\label{interactive-sessions}

To simply enter the container and poke around, use an interactive
session by including \texttt{-it} and overriding the container's
entry-point.

\begin{Shaded}
\begin{Highlighting}[]
\ExtensionTok{docker}\NormalTok{ run -it slapnap/slapnap /bin/bash}
\end{Highlighting}
\end{Shaded}

This will enter you into the container in a bash terminal. This may be
useful for exploring the file structure, examining versions of
\texttt{R} packages that are included in the container, etc.

\section{Examples}\label{sec:examples}

\subsection{\texorpdfstring{Basic calls to
\texttt{slapnap}}{Basic calls to slapnap}}\label{basic-calls-to-slapnap}

A call to \texttt{slapnap} with all default options can be run using the
following command.

\begin{Shaded}
\begin{Highlighting}[]
\ExtensionTok{docker}\NormalTok{ run -v /path/to/local/dir:/home/output slapnap/slapnap}
\end{Highlighting}
\end{Shaded}

Note that this call mounts the local directory
\texttt{path/to/local/dir} to receive output from the container (Section
\ref{sec:mounting}).

When this command is executed,

\subsection{Super learning}\label{super-learning}

\subsection{Train an algorithm}\label{train-an-algorithm}

\subsection{Pull and clean data}\label{pull-and-clean-data}

\section{Report details}\label{sec:report}

\section{Data details}\label{sec:data}

\section{Method details}\label{sec:methods}

\subsection{Outcome definitions}\label{sec:outcomedefs}

\subsection{Super learner details}\label{sec:sldetails}

\subsection{Variable importance
details}\label{variable-importance-details}

\subsubsection{Biological importance}\label{sec:biolimp}

\subsubsection{Predictive importance}\label{predictive-importance}

\section{References}\label{sec:refs}

\bibliography{refs.bib}

\end{document}
